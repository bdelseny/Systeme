\section{Lancement de gs par fork et exec}
\subsection{Vérification de la taille de police et du nombre de lignes}

Afin de vérifier si la taille de la police et si le nombre de ligne sont correctes on peut ajouter le code suivant \ref{lst:verif} dans le code source de \verb!affiche_triangle.c!

\begin{lstlisting}[language=C, label={lst:verif}, caption={Vérification de la taille de police comprise entre 8 et 24, et du nombre de lignes comprit entre 1 et 40.}]
  if(taille_police < 8 || taille_police > 24){
	  fprintf(stderr, "Taille de police incorrecte, valeur entre 8 et 24 attendue\n");
	  exit(1);
  }
  if(nb_lignes < 1 || nb_lignes > MAX_LIGNES){ // MAX_LIGNES = 40
	  fprintf(stderr, "Nombre de lignes incorrecte, valeur entre 1 et %d attendue\n", MAX_LIGNES);
	  exit(1);
  }
\end{lstlisting}

\subsection{Une histoire de père et de fils}
La fonction C \verb!frok()! permet de créer un fils. 
Il faut commencer par initialiser deux variables dans \verb!affiche_triangle.ps!. Une variable qui permettra de récupérer le pid du fils grâce à la commande \verb!pid_t p;!. Il faut aussi une variable qui permettra de récupérer le code de retour, code d'erreur ou de réussite de la fin d'exécution du fils, grâce à la commande \verb!int retour;!.
Afin que le processus fils exécute \verb!gv! ou \verb!gs! il faut procéder comme indiquer dans le code \ref{lst:pereFils}.

\begin{lstlisting}[language=C, label={lst:pereFils}, caption={Création d'un fils qui execute la fonction execve permettant la lecture du fichier PostScript par un interpréteur de PostScript.}]
/* Generation des triangles par le pere */
sortie = "stdout";
taille_triangle = nb_lignes;
postscript_triangle (taille_police);
sleep (3);

/* Creation d'un processus fils */
p = fork();

/* Verification de la creation d'un fils a l'aide de la commande fork */
if(p < 0) {
	fprintf(stderr, "Luke (fils) n'a pas pu etre creer par fork\nVador (pere) est triste\n");
	exit(1);
}

/* Fils creer  plus qu'a attendre qu'il affiche les triangles*/
if(p != 0){
	/* Attendre la terminaison du fils */
	wait(&retour);
	fprintf(stderr, "Generation et affichage du triangle de Pascal termine\n");
} else {	
	/* Execution de l'interpreteur de PostScript */
	execve(nomGV, argumentsGV, envp);
	//execve(nomGS, argumentsGS, envp);
}
\end{lstlisting}