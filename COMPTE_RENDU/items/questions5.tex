\section{Création d'un tube entre triangle et gs}

Le but ici est que le processus père créer un tube et créer deux fils qui intéragissent à l'aide de ce tube.
Ainsi on commence par créer un tube de redirection à l'aide de la commande \verb!pipe()!.
Une fois le tube créé, on crée un processus fils qui va générer le PostScript de triangle vers l'entrée du tube.
Ainsi la sortie standard est redirigée vers l'entrée du tube de redirection.
On termine le processus fils et on créer un nouveau processus fils.\\
On redirige l'entrée standard, qui sera donc celle du nouveau fils, vers la sortie du tube de redirection.
Le processus fils lance alors l'interprète de PostScript.\\
Une fois que l'interpréteur de PostScript est terminé, on termine le processus fils puis on ferme le tube de redirection.\\
L'ensemble du code est détaillé ici : \ref{lst:tubeq5}.

\begin{lstlisting}[language=C, label={lst:tubeq5}, caption={Utilisation d'un tube dans la fonction main.}]
int main(int argc, char *argv[], char *envp[])
{
  /* Initialisation des variables */
  unsigned int taille_police, nb_lignes;
  char nom_executable[200];
  pid_t p_gauche, p_droit;
  int retour_gauche, retour_droit;
  int tube[2];

  lire_args(argc,argv,3,message_usage, 
        "%s",nom_executable,"",
        "%d",&taille_police,"taille_de_police_incorrecte",
        "%d",&nb_lignes,"nombre de lignes incorrect");

  /* Verification des valeurs */
  /* de taille_police [8,24] et nb_lignes [1,MAX_LIGNES]  */
  if(taille_police < 8 || taille_police > 24){
          fprintf(stderr, "Taille de police incorrecte, valeur entre 8 et 24 attendue\n");
          exit(1);
  }
  if(nb_lignes < 1 || nb_lignes > MAX_LIGNES){
          fprintf(stderr, "Nombre de lignes incorrecte, valeur entre 1 et %d attendue\n", MAX_LIGNES);
          exit(1);
  }
    
        /* Creation d'un tube */
        pipe(tube);

        /* Fils gauche */
        p_gauche = fork();
        
        if(p_gauche < 0) { // erreur
                fprintf(stderr, "Le fils n'a pas pu etre cree\n");
        }

		/* Le fils gauche genere le PostScript,
			envoie la sortie standard vers l'entree du tube */
        if(p_gauche == 0){
                close(tube[0]); // fermeture de la sortie du tube
                dup2(tube[1],1);
                sortie = "stdout";
                taille_triangle = nb_lignes;
                postscript_triangle (taille_police);
                sleep (3);
                close(tube[1]);
                exit(0);
                
        }

		p_droit = fork();
        if(p_droit < 0) { // erreur
                fprintf(stderr, "Le fils n'a pas pu etre cree\n");
        }
		
		/* Le fils droit recupere la sortie du tube,
			ouvre l'interprete de PostScript */
        if(p_droit == 0){
                close(tube[1]); // On ferme la sortie du tube
                dup2(tube[0],0);
                execve(nomGV, argumentsGV, envp);
                close(tube[0]);
                exit(0);
        }

        /* fermeture des tubes du pere */    
        close(tube[0]);
        close(tube[1]);

        /* Attente de la fin des fils */
        waitpid(p_gauche, &retour_gauche, 0);
        waitpid(p_droit, &retour_droit, 0);
        
        fprintf(stderr, "Generation et affichage du triangle de Pascal termine\n");
        
        
  return 0;
}
\end{lstlisting}
