\section{Génération du PostScript dans un fichier.}

Après avoir généré le PostScript du triangle dans \verb!triangle.ps! il est possible de connaître la taille de ce fichier PostScript. 
Pour cela il suffit d'effectuer la commande \verb!ls -l triangle.ps! dans le terminal.
On observe que la taille du fichier est de $1,9ko$, appelons $T1$ cette taille.

\subsection{Un problème de taille}

Après avoir supprimé le fichier de $1,9ko$ \verb!triangle.ps! et modifié le code de la procédure main pour affecter la sortie à un fichier \verb!triangle.ps! il faut compiler le code et exécuter le programme.
À la fin de l'exécution il y a un nouveau fichier \verb!triangle.ps!.
Ce fichier fait une taille $T2$ de $185o$. 
La taille de ce nouveau fichier $T2$ est donc largement inférieure à la taille du fichier précédent $T1$.
Lorsque l'on ouvre \verb!triangle.ps! avec un interpréteur de PostScript on se rend compte qu'un seul triangle a été créé.

\subsection{Il nous manque des cases}
\label{sub:cases}

La ligne $66$ de la fonction \verb!boite_chiffre()! du code \verb!boite_chiffre.c! permet de définir les options d'écriture dans le fichier de sortie.
On remarque que le code est le suivant : \verb!fsortie = fopen(sortie, ``w'')!.
L'option \verb!w! permet de créer un nouveau fichier, en supprimant celui existant, et d'écrire dans ce fichier.
Or dans le code \verb!affiche_triangle.c! on observe que la fonction \verb!postscript_triangle()! fait appelle à une boucle sur la fonction \verb!boite_chiffre()!.
Donc à chaque appel de \verb!boite_chiffre()! un nouveau fichier va être créé et une seule boite sera dessinée.
Pour palier à ce problème il nous faut changer l'option \verb!w! de la fonction \verb!fopen()! par l'option \verb!a! qui permet de continuer l'écriture dans le fichier existant si celui-ci existe, sinon de créer le fichier.

\subsection{Un problème de superposition}

Maintenant un nouveau problème fait face.
Lorsque nous lançons plusieurs fois l'exécution du programme les triangles se superposent.
Ce qui rend à la fois la présentation des triangles illisibles et la taille des fichiers exponentielle.
Il faut donc qu'au début de l'exécution du programme un nouveau fichier soit créé afin de ne pas ré-écrire à la suite d'un fichier existant.
Pour ce faire il faut modifier le code de la fonction \verb!postscript_triangle()! du code \verb!affiche_triangle.c!.
Il faut en début de cette fonction initialiser un fichier si la sortie n'est pas la sortie standard.
On rajoute alors le code \ref{lst:sortiePremiere} en début de la fonction \verb!postscript_triangle()!

\begin{lstlisting}[language=C, label={lst:sortiePremiere}, caption={Initialisation du fichier de sortie.}]
if( strcmp(sortie, ``stdout'') != 0 ) {
  fopen(sortie, ``w'');
}
\end{lstlisting}

On remarque que suite à cette modification la taille du fichier sera la même en mettant la sortie vers un fichier dans le code source du programme ou en redirigeant la sortie standard vers un fichier.
Ainsi à la fin de ces modification $T1=T2$.

